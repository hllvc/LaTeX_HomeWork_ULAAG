\documentclass{article}
\usepackage{amsmath, amssymb, graphics, setspace}
\usepackage[width=15cm,top=2cm]{geometry}

\newcommand{\mathsym}[1]{{}}
\newcommand{\unicode}[1]{{}}

\newtheorem{theorem}{\textbf{Zadatak}}
\newenvironment{solution}{\noindent\textbf{Rje\v senje:\newline}}{$\blacksquare$}

\begin{document}

\begin{theorem}
    Ako je
        \(A=\left(
        \begin{array}{ccc}
            2 & 3 & 2 \\
            1 & 2 & -3 \\
            3 & 4 & 1 \\
        \end{array}
        \right)\)
    i
    \(B=\left(
    \begin{array}{c}
        0 \\
        -16 \\
        12 \\
    \end{array}
    \right)\),
    odrediti $A^{-2}B$.
\end{theorem}

\begin{solution}
    Odredimo prvo inverznu matricu matrice $A$, tjst $A^{-1}$.
    Da bi postojala inverzna matrica matrice $A$, determinanta matrice $A$ mora biti razli\v cita od nule ($detA\neq 0$).
    Pa provjerimo prvo determinantu matrice $A$,
    \newline
    \begin{equation*}
        \begin{aligned}
            detA=\left|
        \begin{array}{ccc}
            2 & 3 & 2 \\
            1 & 2 & -3 \\
            3 & 4 & 1 \\
        \end{array}
        \right|
        &=2\cdot \left|
        \begin{array}{cc}
            2 & -3 \\
            4 & 1 \\
        \end{array}
        \right| -
        1\cdot \left|
        \begin{array}{cc}
            3 & 2 \\
            4 & 1 \\
        \end{array}
        \right| +
        3\cdot \left|
        \begin{array}{cc}
            3 & 2 \\
            2 & -3 \\
        \end{array}
        \right|\\
        &=2\cdot 14 -1\cdot (-5)+3\cdot (-13)\\
        &=28+5-39\\
        &=-6=detA\neq 0\Rightarrow\exists A^{-1}.
        \end{aligned}
    \end{equation*}
    Formula za inverznu matricu matrice A je
    $$A^{-1}=\frac{1}{detA}\cdot adjA,$$
    gdje je $adjA=cofA^T$. Odredimo $cofA$.
    \begin{equation*}
        cofA=\left(
        \begin{array}{ccc}
            A_{11} & A_{12} & A_{13} \\
            A_{21} & A_{22} & A_{23} \\
            A_{31} & A_{32} & A_{33} \\
        \end{array}
        \right)
        \end{equation*}.\\
        $A_{ij}$ predstavlja determinantu kada prekrizimo $i-tu$ vrstu i $j-tu$ kolonu matrice $A$, isto tako za svaki \v clan $cofA$ tokom ra\v cunanja
        uzimamo sljede\v ci raspored predznaka,
        \begin{equation*}
            cofA=\left(
            \begin{array}{ccc}
                + & - & + \\
                - & + & - \\
                + & - & + \\
            \end{array}
            \right)
            \end{equation*}.\\
    \begin{align*}
        A_{11}&=\left|
        \begin{array}{cc}
            2 & -3 \\
            4 & 1 \\
        \end{array}
        \right|=2+12=14 &
        A_{12}&=\left|
        \begin{array}{cc}
            1 & -3 \\
            3 & 1 \\
        \end{array}
        \right|=1+9=10 &
        A_{13}&=\left|
        \begin{array}{cc}
            1 & 2 \\
            3 & 4 \\
        \end{array}
        \right|=-2\\
        A_{21}&=3-8=5 & A_{22}&=2-6=-4 & A_{23}&=8-9=-1\\
        A_{31}&=-9-4=-13 & A_{32}&=-6-2=8 & A_{33}&=4-3=1
    \end{align*}
    Sada formirajmo matricu $cofA$,
    \begin{equation*}
        cofA=\left(
        \begin{array}{ccc}
            14 & 5 & -13 \\
            -10 & -4 & 8 \\
            -2 & 1 & 1 \\
        \end{array}
        \right)
        \end{equation*}.\\
        Dakle, $adjA$ je zapravi transponovana $cofA$, tjst
        \begin{equation*}
            adjA=\left(
            \begin{array}{ccc}
                14 & -10 & -2 \\
                5 & -4 & 1 \\
                -13 & 8 & 1 \\
            \end{array}
            \right)^T=\left(
            \begin{array}{ccc}
                14 & 5 & -13 \\
                -10 & -4 & 8 \\
                -2 & 1 & 1 \\
            \end{array}
            \right)
            \end{equation*}.\\
        Dalje,
        \begin{equation*}
            A^{-1}=\frac{1}{-6}\cdot\left(
                \begin{array}{ccc}
                    14 & 5 & -13 \\
                    -10 & -4 & 8 \\
                    -2 & 1 & 1 \\
                \end{array}
                \right)=\left(
                    \begin{array}{ccc}
                     -\frac{7}{3} & -\frac{5}{6} & \frac{13}{6} \\
                     \frac{5}{3} & \frac{2}{3} & -\frac{4}{3} \\
                     \frac{1}{3} & -\frac{1}{6} & -\frac{1}{6} \\
                    \end{array}
                    \right)
            \end{equation*}.\\
\end{solution}

\begin{doublespace}
\noindent\(\left(
\begin{array}{ccc}
 \frac{43}{9} & \frac{37}{36} & -\frac{155}{36} \\
 -\frac{29}{9} & -\frac{13}{18} & \frac{53}{18} \\
 -\frac{10}{9} & -\frac{13}{36} & \frac{35}{36} \\
\end{array}
\right)\)
\end{doublespace}

\begin{doublespace}
\noindent\(\left(
\begin{array}{c}
 -\frac{613}{9} \\
 \frac{422}{9} \\
 \frac{157}{9} \\
\end{array}
\right)\)
\end{doublespace}

\end{document}
