\documentclass{article}
\usepackage{amsmath, amssymb, graphics, setspace, cancel, mathtools}
\usepackage[T1]{fontenc}
\usepackage[width=15cm,top=2cm]{geometry}

\newcommand{\com}[1]{\tag{\textrm{\footnotesize #1}}}
\newtheorem{theorem}{\textbf{Zadatak}}
\newenvironment{solution}{\noindent\textbf{Rje\v senje:\newline}}{$\hfill\spadesuit$}

\begin{document}

\begin{titlepage}
    \begin{center}
        \vspace*{1cm}

        \huge
        \textbf{Zada\' ca I}

        \vspace{0.5cm}
        \large
        Uvod u linearnu algebru i analiti\v cka geometrija
            
        \vspace{1.5cm}

        \textbf{Halilovi\' c Adis}

        \vfill
            
        % A thesis presented for the degree of\\
        % Doctor of Philosophy
            
        % \vspace{0.8cm}
        \large
        Prirodno-matemati\v cki univerzitet u Tuzli\\
        Odsijek matematika\\
        Bosna i Hercegovina\\
        11. Jul 2020.
    \end{center}
\end{titlepage}

% Zadatak 1
\begin{theorem}
    Ako je
        \(A=\left(
        \begin{array}{ccc}
            2 & 3 & 2 \\
            1 & 2 & -3 \\
            3 & 4 & 1 \\
        \end{array}
        \right)\)
    i
    \(B=\left(
    \begin{array}{c}
        0 \\
        -16 \\
        12 \\
    \end{array}
    \right)\),
    odrediti $A^{-2}B$.
\end{theorem}

\begin{solution}
    Odredimo prvo inverznu matricu matrice $A$, tjst $A^{-1}$.
    Da bi postojala inverzna matrica matrice $A$, determinanta matrice $A$ mora biti razli\v cita od nule ($detA\neq 0$).
    Pa provjerimo prvo determinantu matrice $A$,
    \newline
    \begin{equation*}
        \begin{aligned}
            detA=\left|
            \begin{array}{ccc}
                2 & 3 & 2 \\
                1 & 2 & -3 \\
                3 & 4 & 1 \\
            \end{array}
            \right|
            &=2\cdot \left|
            \begin{array}{cc}
                2 & -3 \\
                4 & 1 \\
            \end{array}
            \right| -
            1\cdot \left|
            \begin{array}{cc}
                3 & 2 \\
                4 & 1 \\
            \end{array}
            \right| +
            3\cdot \left|
            \begin{array}{cc}
                3 & 2 \\
                2 & -3 \\
            \end{array}
            \right|\\
            &=2\cdot 14 -1\cdot (-5)+3\cdot (-13)\\
            &=28+5-39\\
            &=-6=detA\neq 0\Rightarrow\exists A^{-1}.
        \end{aligned}
    \end{equation*}
    Formula za inverznu matricu matrice A je
    $$A^{-1}=\frac{1}{detA}\cdot adjA,$$
    gdje je $adjA=cofA^T$. Odredimo $cofA$.
    \begin{equation*}
        cofA=\left(
        \begin{array}{ccc}
            A_{11} & A_{12} & A_{13} \\
            A_{21} & A_{22} & A_{23} \\
            A_{31} & A_{32} & A_{33} \\
        \end{array}
        \right)
    \end{equation*}.\\
    $A_{ij}$ predstavlja determinantu kada prekrizimo $i-tu$ vrstu i $j-tu$ kolonu matrice $A$, isto tako za svaki \v clan $cofA$ tokom ra\v cunanja
    uzimamo sljede\v ci raspored predznaka,
    \begin{equation*}
        cofA=\left(
        \begin{array}{ccc}
            + & - & + \\
            - & + & - \\
            + & - & + \\
        \end{array}
        \right)
    \end{equation*}.\\
    \begin{align*}
        A_{11}&=\left|
        \begin{array}{cc}
            2 & -3 \\
            4 & 1 \\
        \end{array}
        \right|=2+12=14 &
        A_{12}&=\left|
        \begin{array}{cc}
            1 & -3 \\
            3 & 1 \\
        \end{array}
        \right|=1+9=10 &
        A_{13}&=\left|
        \begin{array}{cc}
            1 & 2 \\
            3 & 4 \\
        \end{array}
        \right|=-2\\
        A_{21}&=3-8=5 & A_{22}&=2-6=-4 & A_{23}&=8-9=-1\\
        A_{31}&=-9-4=-13 & A_{32}&=-6-2=8 & A_{33}&=4-3=1
    \end{align*}
    Sada formirajmo matricu $cofA$,
    \begin{equation*}
        cofA=\left(
        \begin{array}{ccc}
            14 & 5 & -13 \\
            -10 & -4 & 8 \\
            -2 & 1 & 1 \\
        \end{array}
        \right)
    \end{equation*}.\\
    Dakle, $adjA$ je zapravi transponovana $cofA$, tjst
    \begin{equation*}
        adjA=\left(
        \begin{array}{ccc}
            14 & -10 & -2 \\
            5 & -4 & 1 \\
            -13 & 8 & 1 \\
        \end{array}
        \right)^T=\left(
        \begin{array}{ccc}
            14 & 5 & -13 \\
            -10 & -4 & 8 \\
            -2 & 1 & 1 \\
        \end{array}
        \right)
    \end{equation*}.\\
    Dalje,
    \begin{equation*}
        A^{-1}=\frac{1}{-6}\cdot\left(
        \begin{array}{ccc}
            14 & 5 & -13 \\
            -10 & -4 & 8 \\
            -2 & 1 & 1 \\
        \end{array}
        \right)=\left(
        \begin{array}{ccc}
            -\frac{7}{3} & -\frac{5}{6} & \frac{13}{6} \\
            \frac{5}{3} & \frac{2}{3} & -\frac{4}{3} \\
            \frac{1}{3} & -\frac{1}{6} & -\frac{1}{6} \\
        \end{array}
        \right)
    \end{equation*}.\\
    Odredimo sada $A^{-2}=(A^{-1})^2$, tjst
    \begin{doublespace}
        \begin{equation*}
            \begin{aligned}
                (A^{-1})^2&=A^{-1}\cdot A^{-1}\\
                &=\left(
                \begin{array}{ccc}
                    -\frac{7}{3} & -\frac{5}{6} & \frac{13}{6} \\
                    \frac{5}{3} & \frac{2}{3} & -\frac{4}{3} \\
                    \frac{1}{3} & -\frac{1}{6} & -\frac{1}{6} \\
                \end{array}
                \right)\cdot \left(
                \begin{array}{ccc}
                    -\frac{7}{3} & -\frac{5}{6} & \frac{13}{6} \\
                    \frac{5}{3} & \frac{2}{3} & -\frac{4}{3} \\
                    \frac{1}{3} & -\frac{1}{6} & -\frac{1}{6} \\
                \end{array}
                \right)\\
                &=\left(
                    \begin{array}{ccc}
                        (-\frac{7}{3})\cdot (-\frac{7}{3})+(-\frac{5}{6})\cdot \frac{5}{3}+\frac{13}{6}\cdot \frac{1}{3} & \frac{37}{36} & -\frac{155}{36} \\
                        \frac{5}{3}\cdot (-\frac{7}{3})+\frac{2}{3}\cdot \frac{5}{3}+(-\frac{4}{3})\cdot \frac{1}{3} & -\frac{13}{18} & \frac{53}{18} \\
                        \frac{1}{3}\cdot (-\frac{7}{3})+(-\frac{1}{6})\cdot \frac{5}{3}+(-\frac{1}{6})\cdot \frac{1}{3} & -\frac{13}{36} & \frac{35}{36} \\
                    \end{array}
                \right)\\
                &=\left(
                    \begin{array}{ccc}
                        \frac{49}{9}-\frac{25}{18}+\frac{13}{18} & \frac{37}{36} & -\frac{155}{36} \\
                        -\frac{35}{9}+\frac{10}{9}-\frac{4}{9} & -\frac{13}{18} & \frac{53}{18} \\
                        -\frac{7}{9}-\frac{5}{18}-\frac{1}{18} & -\frac{13}{36} & \frac{35}{36} \\
                    \end{array}
                \right)\\
                A^{-2}&=\left(
                \begin{array}{ccc}
                    \frac{43}{9} & \frac{37}{36} & -\frac{155}{36} \\
                    -\frac{29}{9} & -\frac{13}{18} & \frac{53}{18} \\
                    -\frac{10}{9} & -\frac{13}{36} & \frac{35}{36} \\
                \end{array}
                \right)
            \end{aligned}
        \end{equation*}
    \end{doublespace}
    Na kraju, pomno\v zimo $A^{-2}$ sa $B$.
    \begin{doublespace}
        \begin{equation*}
            \begin{aligned}
                A^{-2}\cdot B &= \left(
                    \begin{array}{ccc}
                        \frac{43}{9} & \frac{37}{36} & -\frac{155}{36} \\
                        -\frac{29}{9} & -\frac{13}{18} & \frac{53}{18} \\
                        -\frac{10}{9} & -\frac{13}{36} & \frac{35}{36} \\
                    \end{array}
                \right)\cdot \left(
                    \begin{array}{c}
                        0 \\
                        -16 \\
                        12 \\
                    \end{array}
                \right)\\
                &= \left(
                    \begin{array}{c}
                        \frac{43}{9}\cdot 0+\frac{37}{36}\cdot (-16)+(-\frac{155}{36})\cdot 12 \\
                        (-\frac{29}{9})\cdot 0+(-\frac{13}{18})\cdot (-16)+\frac{53}{18}\cdot 12 \\
                        (-\frac{10}{9})\cdot 0+(-\frac{13}{36})\cdot (-16)+\frac{35}{36}\cdot 12 \\
                    \end{array}
                \right)\\
                &= \left(
                    \begin{array}{c}
                        -\frac{613}{9} \\
                        \frac{422}{9} \\
                        \frac{157}{9} \\
                    \end{array}
                \right)
            \end{aligned}
        \end{equation*}
    \end{doublespace}
\end{solution}

% Zadatak 2
\begin{theorem}
    Ne razvijaju\' ci determinante pokazati da je
    \begin{equation}
        \label{eq:1}
        \left|
        \begin{array}{cccc}
            0 & a & b & c \\
            a & 0 & c & b \\
            b & c & 0 & a \\
            c & b & a & 0 \\
        \end{array}
        \right|=\left|
        \begin{array}{cccc}
            0 & 1 & 1 & 1 \\
            1 & 0 & c^2 & b^2 \\
            1 & c^2 & 0 & a^2 \\
            1 & b^2 & a^2 & 0 \\
        \end{array}
        \right|.
    \end{equation}
\end{theorem}

\begin{solution}
    Pomno\v zimo prvo drugu kolonu sa $bc$, tre\' cu sa $ac$ i \v cetvrtu sa $ab$.
    Kako je $bc\cdot ac\cdot ab=a^2b^2c^2$, cijelu determinantu moramo pomno\v ziti sa $\frac{1}{a^2b^2c^2}$.
    Zatim pomno\v zimo drugu, tre\' cu i \v cetvrtu vrstu redom sa $a$, $b$ i $c$ ($a\cdot b\cdot c=abc\cdot \frac{1}{a^2b^2c^2}=\frac{abc}{a^2b^2c^2})$.
    Nakon kra\' cenja razlomka, kada pomno\v zimo prvu vrstu determinante sa $\frac{1}{abc}$ dobijamo jendakost i vrijedi (\ref{eq:1}).

    \begin{equation*}
        \begin{aligned}
        \left|
            \begin{array}{cccc}
                0 & a & b & c \\
                a & 0 & c & b \\
                b & c & 0 & a \\
                c & b & a & 0 \\
            \end{array}
        \right|&=\frac{1}{a^2b^2c^2}\cdot\left|
            \begin{array}{cccc}
                0 & abc & abc & abc \\
                a & 0 & ac^2 & ab^2 \\
                b & bc^2 & 0 & a^2b \\
                c & b^2c & a^2c & 0 \\
            \end{array}
        \right|\\
        &=\frac{\cancelto{1}{abc}}{a^{\cancel{2}}b^{\cancel{2}}c^{\cancel{2}}}\cdot \left|
            \begin{array}{cccc}
                0 & abc & abc & abc \\
                1 & 0 & c^2 & b^2 \\
                1 & c^2 & 0 & a^2 \\
                1 & b^2 & a^2 & 0 \\
            \end{array}
        \right|\\
        &=\frac{1}{\cancelto{1}{abc}}\cdot \left|
        \begin{array}{cccc}
            0 & \cancelto{1}{abc} & \cancelto{1}{abc} & \cancelto{1}{abc} \\
            1 & 0 & c^2 & b^2 \\
            1 & c^2 & 0 & a^2 \\
            1 & b^2 & a^2 & 0 \\
        \end{array}
        \right|\\
        &=\left|
            \begin{array}{cccc}
                0 & 1 & 1 & 1 \\
                1 & 0 & c^2 & b^2 \\
                1 & c^2 & 0 & a^2 \\
                1 & b^2 & a^2 & 0 \\
            \end{array}
        \right|
        \end{aligned}
    \end{equation*}
\end{solution}

% Zadatak 3
\begin{theorem}
    Ako je $M=\left(
        \begin{array}{ccc}
            1 & 0 & 0 \\
            1 & 0 & 1 \\
            0 & 1 & 0 \\
        \end{array}
    \right)$
    dokazati da je
    \begin{equation}
        \label{eq:2}
        M^n=M^{n-2}+M^2-I\ za\ (n\in \mathbb{N}, n\geq 3)
    \end{equation}
\end{theorem}

\begin{solution}
    \begin{itemize}
        \item Provjerimo prvo da li vrijedi za $n=3$.
        \begin{equation*}
            \begin{aligned}
                M^3 &= M^2\cdot M \\
                    &= M\cdot M\cdot M \\
                    &= \left(
                        \begin{array}{ccc}
                            1 & 0 & 0 \\
                            1 & 0 & 1 \\
                            0 & 1 & 0 \\
                        \end{array}
                        \right)\cdot\left(
                        \begin{array}{ccc}
                            1 & 0 & 0 \\
                            1 & 0 & 1 \\
                            0 & 1 & 0 \\
                        \end{array}
                        \right)\cdot\left(
                        \begin{array}{ccc}
                            1 & 0 & 0 \\
                            1 & 0 & 1 \\
                            0 & 1 & 0 \\
                        \end{array}
                        \right)\\
                    &= \left(
                        \begin{array}{ccc}
                            1 & 0 & 0 \\
                            1 & 1 & 0 \\
                            1 & 0 & 1 \\
                        \end{array}
                        \right)\cdot\left(
                        \begin{array}{ccc}
                            1 & 0 & 0 \\
                            1 & 0 & 1 \\
                            0 & 1 & 0 \\
                        \end{array}
                        \right)\\
                    &= \left(
                        \begin{array}{ccc}
                            1 & 0 & 0 \\
                            2 & 0 & 1 \\
                            1 & 1 & 0 \\
                        \end{array}
                    \right)
            \end{aligned}
        \end{equation*}
        \begin{equation}
            \label{eq:3}
            \begin{aligned}
                M^3 &= M^{3-2}+M^2-I\\
                    &= M^1+M^2-I\\
                    &= \left(
                    \begin{array}{ccc}
                        1 & 0 & 0 \\
                        1 & 0 & 1 \\
                        0 & 1 & 0 \\
                    \end{array}
                    \right)+\left(
                    \begin{array}{ccc}
                        1 & 0 & 0 \\
                        1 & 0 & 1 \\
                        0 & 1 & 0 \\
                    \end{array}
                    \right)^2-\left(
                    \begin{array}{ccc}
                        1 & 0 & 0 \\
                        0 & 1 & 0 \\
                        0 & 0 & 1 \\
                    \end{array}
                    \right)\\
                    &= \left(
                    \begin{array}{ccc}
                        1 & 0 & 0 \\
                        1 & 0 & 1 \\
                        0 & 1 & 0 \\
                    \end{array}
                    \right)+\left(
                    \begin{array}{ccc}
                        1 & 0 & 0 \\
                        1 & 1 & 0 \\
                        1 & 0 & 1 \\
                    \end{array}
                    \right)-\left(
                    \begin{array}{ccc}
                        1 & 0 & 0 \\
                        0 & 1 & 0 \\
                        0 & 0 & 1 \\
                    \end{array}
                    \right)\\
                    &= \left(
                        \begin{array}{ccc}
                            2 & 0 & 0 \\
                            2 & 1 & 1 \\
                            1 & 1 & 1 \\
                        \end{array}
                    \right)-\left(
                        \begin{array}{ccc}
                            1 & 0 & 0 \\
                            0 & 1 & 0 \\
                            0 & 0 & 1 \\
                        \end{array}
                    \right)\\
                    &= \left(
                        \begin{array}{ccc}
                            1 & 0 & 0 \\
                            2 & 0 & 1 \\
                            1 & 1 & 0 \\
                        \end{array}
                    \right)
            \end{aligned}
        \end{equation}
        Jasno vidimo da vrijedi (\ref{eq:2}) za $n=3$.
        \item Pretpostavimo da (\ref{eq:2}) vrijedi i za neko $n\in\mathbb{N}$ takvo da je $n\geq 3$.
        \begin{equation}
            \label{eq:ih}
            M^n=M^{n-2}+M^2-I\com{I.H.}
        \end{equation}
        \item Poka\v zimo da je tvrdnja ta\v cna i za $n+1$, tjst da vrijedi
        $$M^{n+1}=M^{n-1}+M^2-I$$
        \begin{equation*}
            \begin{aligned}
                M^{n+1}=M^nM    &\overset{\ref{eq:ih}}{=} (M^{n-2}+M^2-I)M \\
                                &= M^{n-2}M+M^2M-IM \\
                                &= M^{n-1}+M^3-M=M^{n-1}+M^2-I \\
                                &\Rightarrow M^3-M=M^2-I \\
                                &= M^3=M+M^2-I
            \end{aligned}
        \end{equation*}
        \v Sto smo ve\' c pokazali u (\ref{eq:3}) da vrijedi.
    \end{itemize}
    Na osnovu potpune matemati\v cke indukcije zaklju\v cujemo da (\ref{eq:2}) vrijedi za svako $n\in\mathbb{N}$ za $n\geq 3$.
\end{solution}

% Zadatak 4
\begin{theorem}
    Odrediti $x$ tako da je
    $$\text{rang}\left[
        \begin{array}{cccc}
            4 & 4 & -3 & 1 \\
            1 & 1 & -1 & 0 \\
            x & 2 & 2 & 2 \\
            9 & 9 & x & 3 \\
        \end{array}
    \right]=3.$$
\end{theorem}

\begin{solution}
    Prvo zamijenimo mjesta tre\' coj i \v cetvrtoj koloni a zati prvoj i tre\' coj koloni.
    \begin{equation}
        \label{eq:4}
        \begin{aligned}
            \left[
        \begin{array}{cccc}
            1 & 4 & 4 & -3 \\
            0 & 1 & 1 & -1 \\
            2 & 2 & x & 2 \\
            3 & 9 & 9 & x \\
        \end{array}
        \right] &\overset{\begin{array}{c}
            III_v=(-2)\cdot I_v+III_v \\
            IV_v=(-3)\cdot I_v+IV_v \\
        \end{array}}{\sim}\left[
            \begin{array}{cccc}
                1 & 4 & 4 & -3 \\
                0 & 1 & 1 & -1 \\
                0 & -6 & -8+x & 8 \\
                0 & 3 & 3 & 9+x \\
            \end{array}
        \right]\\
        &\overset{\begin{array}{c}
            III_v=6\cdot II_v+IIIv \\
            IV_v=(-3)\cdot II_v+IV_v \\
        \end{array}}{\sim}\left[
            \begin{array}{cccc}
                1 & 4 & 4 & -3 \\
                0 & 1 & 1 & -1 \\
                0 & 0 & -2+x & 2 \\
                0 & 0 & 0 & 6+x \\
            \end{array}
        \right]
        \end{aligned}
    \end{equation}
    \begin{equation*}
        \begin{aligned}
            \text{rang}\left[
            \begin{array}{cccc}
                1 & 4 & 4 & -3 \\
                0 & 1 & 1 & -1 \\
                0 & 0 & -2+x & 2 \\
                0 & 0 & 0 & 6+x \\
            \end{array}
            \right]=3&\Leftrightarrow 6+x=0\\
                    &\Leftrightarrow x=-6
        \end{aligned}
    \end{equation*}
    Za $x=-6$ $\text{rang}\left[
        \begin{array}{cccc}
            4 & 4 & -3 & 1 \\
            1 & 1 & -1 & 0 \\
            x & 2 & 2 & 2 \\
            9 & 9 & x & 3 \\
        \end{array}
    \right]=3$.
\end{solution}

% Zadatak 5
\begin{theorem}
    Odrediti one vrijednosti $\lambda\in\mathbb{R}$ za koje sistem
    \begin{equation}
        \label{eq:5}
        \begin{aligned}
             x\ \ &+&y\ \ &+&2z\ \ &+&t&\ \ =&-1 \\
            -x\ \ &+&y\ \ &-&3z\ \ &+&\lambda t&\ \ =&1 \\
             x\ \ &+&y\ \ &+& z\ \ &+& &\ \ =&2 \\
             x\ \ &+& \ \ & &2z\ \ & & &\ \ =&\lambda +1
        \end{aligned}
    \end{equation}
    ima rje\v senje.
\end{theorem}

\begin{solution}
    Matrica $A|b=\left[
        \begin{array}{cccc|c}
            1 & 1 & 2 & 1 & -1 \\
            -1 & 1 & -3 & \lambda & 1 \\
            1 & 1 & 1 & 0 & 2 \\
            1 & 0 & 2 & 0 & \lambda +1 \\
        \end{array}
    \right]$ predstavlja pro\v sirenu matricu sistema (\ref{eq:5}).\\
    Koriste\' ci elementarne transformacije matrice radimo sljede\' ce
    \begin{equation*}
        \begin{aligned}
            \left[
                \begin{array}{cccc|c}
                    1 & 1 & 2 & 1 & -1 \\
                    -1 & 1 & -3 & \lambda & 1 \\
                    1 & 1 & 1 & 0 & 2 \\
                    1 & 0 & 2 & 0 & \lambda +1 \\
                \end{array}
            \right]\overset{\begin{array}{c}
                II_v=I_v+II_v \\
                III_v=(-1)\cdot I_v+III_v \\
                IV_v=(-1)\cdot I_v+IV_v \\
            \end{array}}{\sim}&\left[
                \begin{array}{cccc|c}
                    1 & 1 & 2 & 1 & -1 \\
                    0 & 2 & -1 & \lambda + 1 & 0 \\
                    0 & 0 & -1 & -1 & 3 \\
                    0 & -1 & 0 & -1 & \lambda + 2 \\
                \end{array}
            \right]\\
            \overset{\begin{array}{c}
                IV_v=II_v+2+IV_v
            \end{array}}{\sim}&\left[
                \begin{array}{cccc|c}
                    1 & 1 & 2 & 1 & -1 \\
                    0 & 2 & -1 & \lambda + 1 & 0 \\
                    0 & 0 & -1 & -1 & 3 \\
                    0 & 0 & -1 & \lambda-1 & 2(\lambda + 2) \\
                \end{array}
            \right]\\
            \overset{\begin{array}{c}
                IV_v=(-1)\cdot III_v+IV_v
            \end{array}}{\sim}&\left[
                \begin{array}{cccc|c}
                    1 & 1 & 2 & 1 & -1 \\
                    0 & 2 & -1 & \lambda + 1 & 0 \\
                    0 & 0 & -1 & -1 & 3 \\
                    0 & 0 & 0 & \lambda & 2\lambda +1 \\
                \end{array}
            \right]=A|b
        \end{aligned}
    \end{equation*}
    Posmatrajmo dobivenu matricu $A|b$. Rang same matrice $A$ bi\' ce 3 ako i samo ako $\lambda$ bude jednaka nula.
    U tom slu\v caju rang matrice $A|b$ biti \' ce jednak \v cetiri. Prema tome $r(A)<r(A|b)$. Prema Kronecker-Capellijevom toremu ovaj sistem nije saglasan,
    odakle slijedi da za $\lambda =0$ sistem (\ref{eq:5}) nema rje\v senje. Za bilo koje $\lambda\neq 0$ sistem (\ref{eq:5}) ima jedinstveno rje\v senje jer
    \' ce $r(A)=r(A|b)=max(m,n)$ gdje su $m$ i $n$ redom broj vrsta i broj kolona matrice $A|b$, odnosno redom broj jedna\v cina i broj nepoznatih.\\
    Pronađimo sada uređenu n-torku, ra\v cunamo 'odozdo na gore'.
    \begin{equation*}
        \begin{aligned}
            \lambda t&=2\lambda +1|\cdot\frac{1}{\lambda} \\
                    \Aboxed{t&=\frac{2\lambda +1}{\lambda}} \\
                    -z&=3+t|\cdot (-1)\\
                    \Aboxed{z&=-(3+t)} \\
                    2y&=z-(\lambda +1)t|\cdot\frac{1}{2} \\
                    \Aboxed{y&=\frac{z-(\lambda +1)t}{2}} \\
                    \Aboxed{x&=-1-y-2z-t} \\
        \end{aligned}
    \end{equation*}
    Dakle za proizvoljno $\lambda\in\mathbb{R}$ imamo sljede\' cu uređenu n-torku.
    $$(x,\ y,\ z,\ t)=\biggl(-1-y-2z-t,\ \frac{z-(\lambda +1)t}{2},\ -(3+t),\ \frac{2\lambda +1}{\lambda}\biggr).$$
    Provjerimo za $\lambda =1$. \\
    \begin{equation*}
        \begin{aligned}
            \boxed{t}&=\frac{2+1}{1}=\boxed{3}, \\
            \boxed{z}&=-(3+3)=\boxed{-6}, \\
            \boxed{y}&=\frac{-6-(1+1)3}{2}=\frac{-6-6}{2}=\boxed{-6}, \\
            \boxed{x}&=-1-(-6)-2(-6)-3=-1+6+12-3=\boxed{14}
        \end{aligned}
    \end{equation*}
    Dakle,
    $$(x,\ y,\ z,\ t)=(14,\ -6,\ -6,\ 3)\land\lambda =1.$$
    Unesimo vrijdnosti u sistem (\ref{eq:5}),
    \begin{equation*}
        \begin{aligned}
            14\ \ &+&(-6)\ \ &+&2(-6)\ \ &+&3&\ \ =&-1 \\
            -14\ \ &+&(-6)\ \ &-&3(-6)\ \ &+&3&\ \ =&1 \\
             14\ \ &+&(-6)\ \ &+& (-6)\ \ &+& &\ \ =&2 \\
             14\ \ &+& \ \ & &2(-6)\ \ & & &\ \ =&1+1 \\\\
             \hline\\
             14\ \ &-&6\ \ &-&12\ \ &+&3&\ \ =&-1 \\
            -14\ \ &-&6\ \ &+&18\ \ &+&3&\ \ =&1 \\
             14\ \ &-&6\ \ &-&6\ \ && &\ \ =&2 \\
             14\ \ &-& \ \ & &12\ \ & & &\ \ =&2 \\\\
             \hline\\
             \ \ & & \ \ & & \ \ -&1& &\ \ =&-1 \\
             \ \ & & \ \ & & \ \ &1& &\ \ =&1 \\
             \ \ & & \ \ & & \ \ &2& &\ \ =&2 \\
             \ \ & & \ \ & & \ \ &2& &\ \ =&2 \\
            \end{aligned}
        \end{equation*}
        
\end{solution}

\end{document}
